\documentclass{beamer}	
\mode<presentation>
 
\usepackage{pdfpages}
\usepackage{fancyvrb}
\usepackage{chemarr}

\usepackage{amsmath}		%% mathematics typesetting
\usepackage{amssymb}
 
\usepackage{epigraph}   %% nice setting of quotations

\usepackage{tabularx} %% allows to use row colours in tables

\usepackage{ulem}

\usepackage{booktabs}

\usepackage{siunitx} %% tpyeset SI units

\usepackage{CJKutf8} %% typeset Chinese characters

\usepackage{pdfpages}%% include pdfs


\usepackage{animate} %% show animated gifs

\DeclareMathAlphabet{\mathcalligra}{T1}{calligra}{m}{n}


% Color and Theme. Can be changed. However, this one's quite nice.
\usetheme{Madrid}
\definecolor{theme}{rgb}{0.84,0,0.21}
\usecolortheme[named=theme]{structure}


%%  Title information
\title[M11.10 Hormone, Regelkreise]{M11.10 Hormone und Regelkreise}
\author[melanie.stefan@medicalschool-berlin.de]{M11.10 Hormone und Regelkreise}
\institute[]{Prof. Melanie Stefan - melanie.stefan@medcialschool-berlin.de}
\date{WiSe 2022/23}
 

% Table of contents to pop up at the beginning of each section
\AtBeginSection[]
{
  \begin{frame}<beamer>
    \frametitle{Outline}
    \tableofcontents[currentsection,currentsubsection]
  \end{frame}
}
 
\beamertemplatenavigationsymbolsempty

\begin{document}


  { \usebackgroundtemplate{\includegraphics[width=1.2\paperwidth]{MSB_Titelseite.pdf}} 
\begin{frame}

 \maketitle 

$\,$\\[6cm] 


\end{frame} 
}






%% Hook: 
\begin{frame}

\begin{columns}[c]
\begin{column}{5cm}
\includegraphics[width=\textwidth]{/home/melanie/Work/pictures/metaphore/howareyou.jpg}

\end{column}

\pause

\begin{column}{5cm}
Hormone bestimmen:

\begin{itemize}
\item
Wie es uns geht
\item
Wie wir existieren
\end{itemize}

\end{column}
\end{columns}

\end{frame}



%% %% TLIA

{
  \usebackgroundtemplate{\includegraphics[width=1.1\paperwidth]{/home/melanie/Work/pictures/physiology/insulin_kit.jpg}}
\begin{frame}

$\,$\\[5cm]

\Large{In dieser Vorlesung geht es um \\Hormone, und wie sie reguliert werden}


\end{frame}
}


%% %% Learning Objectives
 
\begin{frame}

\frametitle{Nach dieser Vorlesung sollten Sie folgendes können}



\begin{block}{Grundlagen:}
\begin{itemize}
\item
den allgemeinen Aufbau hormoneller Regelkreise erklären und Beispiele geben
\item
Hormone unterteilen 
\begin{itemize}
\item
 nach Wirkungsradius
\item 
nach chemischen Eigenschaften
\end{itemize}
\item
das hypothalamisch-hypophysäre System beschreiben und erklären
\item
Hormone des hypothalamisch-hypophysären Systems aufzählen und ihre Aktivierungsmuster beschreiben
\item
die Wirkung, chemischen Eigenschaften und Regulierung von T3 und T4 beschreiben.
\item
Hormone der Nebennierenrinde klassifizieren und ihre Regulierung und Wirkung beschreiben 
\item
Regulierung und Wirkung von Insulin und Glukagon erklären

\end{itemize}

\end{block}


\end{frame}



\begin{frame}

\frametitle{Nach dieser Vorlesung sollten Sie folgendes können}


\begin{block}{Klinik:}
\begin{itemize}
\item
Ursachen von Hormonstörungen benennen
\item
 Erkrankungen der Schilddrüse beschreiben und erklären
\item
 Morbus Cushing beschreiben und erklären
\item
Diabetes mellitus Typ 1 und 2 unterscheiden und erklären 
\end{itemize}

\end{block}

\end{frame}


%% %% %% Main Body

\section{Insulin und Glukagon}


%% Review: Zuckerhaushalt
\begin{frame}
\frametitle{Kleine Erinnerung}

\begin{columns}[c]

\begin{column}{5cm}
\begin{center}
\includegraphics[width=\textwidth]{/home/melanie/Work/pictures/physiology/zuckerln.jpg}
\end{center}
\end{column}

\begin{column}{5cm}

Wie wird Glukose im Körper \dots

\begin{itemize}
\item
Verwendet?
\item
Gespeichert?
\item
Bei Bedarf aus Speichern freigesetzt?
\item
Hergestellt?
\end{itemize}


\end{column}


\end{columns}



\end{frame}



\begin{frame}
\frametitle{Kleine Erinnerung}

\begin{columns}[c]

\begin{column}{5cm}

\begin{center}
\includegraphics[width=\textwidth]{/home/melanie/Work/pictures/physiology/zuckerln.jpg}
\end{center}

\end{column}

\begin{column}{5cm}

Wie wird Glukose im Körper \dots

\begin{itemize}
\item
Verwendet? \\ \textcolor{theme}{- Glykolyse zur Gewinnung von Energie}
\item
Gespeichert? \\ \textcolor{theme}{- Glykogensynthese}
\item
Bei Bedarf aus Speichern freigesetzt? \\  \textcolor{theme}{- Glykogenolyse}
\item
Hergestellt? \\ \textcolor{theme}{- Glukoneogenese aus Aminosäuren und Lipiden}
\end{itemize}

\pause
Diese Vorgänge werden hormonell kontrolliert

\end{column}

\end{columns}

\end{frame}



\begin{frame}
\frametitle{Insulin und Glukagon}

%% Bild: Chemie, text: peptidhormon, langerhansdings
\begin{columns}[t]
\begin{column}{5cm}

Insulin:

\begin{center}
\includegraphics[width=0.8\textwidth]{/home/melanie/Work/pictures/physiology/insulin.png}
\end{center}
$\,$\\


Polypeptidhormon \\[0.2 cm]

Gebildet in den \textcolor{theme}{\textbf{B}}-Zellen der Pankreas \\
(in den Langerhans-Inseln)

\end{column}

\begin{column}{5cm}

Glukagon:

\begin{center}
\includegraphics[width=\textwidth]{/home/melanie/Work/pictures/physiology/glucagon.jpg}
\end{center}
$\,$\\[0.4 cm]


Polypeptidhormon \\[0.2 cm]

Gebildet in den \textcolor{theme}{\textbf{A}}-Zellen der Pankreas \\
(in den Langerhans-Inseln)

\end{column}
\end{columns}

\end{frame}




%% Wirkung von Insulin
\begin{frame}
\frametitle{Insulin sorgt für die Verarbeitung von Glukose}

Insulin fördert die Aufnahme von Glukose in Zellen, die Glykolyse und die Speicherung von Glukose.

\begin{center}
\includegraphics[width=0.6\textwidth]{/home/melanie/Work/pictures/physiology/Insulin_Wirkung.png}
\end{center}



\end{frame}



%% Ausschüttung von Insulin
\begin{frame}
\frametitle{Glukose stimuliert die Freisetzung von Insulin}

\begin{center}
\includegraphics[width=\textwidth]{/home/melanie/Work/pictures/physiology/Insulinfreisetzung_schematisch.png}
\end{center}


\end{frame}


%% Wirkung von Glukagon
\begin{frame}
\frametitle{Aber was ist, wenn wir Energie brauchen?}

\pause

\begin{itemize}
\item
Freisetzung von Glukagon aus den A-Zellen der Pankreas
\item
  The clue is in the name! \textbf{gluc}ose \textbf{agon}ist 
\item
Glukagon fördert Freisetzung von Glukose aus Glykogen (Glykogenolyse) und die Bildung von neuer Glukose (Glukoneogenese)
\end{itemize}
\end{frame}




\begin{frame}
\frametitle{Glukagon wirkt über einen G-Protein-gekoppelten Rezeptor}

\begin{center}
\includegraphics[width=0.7\textwidth]{/home/melanie/Work/pictures/physiology/glucagon_pathway.png}
\end{center}


\end{frame}

%%%%%%%%%%%%%%%%%%&%%%%
%% compiled to here
%%%%%%%%%%%%%%%%%%&%%%%


%% Wechselwirkung Glukagon, Insulin
\begin{frame}
\frametitle{Zusammenfassung}

\begin{center}
\includegraphics[width=0.8\textwidth]{/home/melanie/Work/pictures/physiology/insulin_glukagon.png}
\end{center}


\end{frame}


%% Diabetes
\begin{frame}
\frametitle{Was kann schiefgehen?}

Diabetes mellitus ist ein Insulinmangel. 

\pause

Exkurs Medizingeschichte: Schon vor 3500 Jahren konnten medizinische Gelehrte in Indien Diabetes mellitus mit Hilfe von Ameisen diagnostizieren. \pause \textcolor{theme}{Wie?}

\begin{center}
\includegraphics[width=0.6\textwidth]{/home/melanie/Work/pictures/animals/ant.jpg}
\end{center}



%% Think-pair-share: ANTS


\end{frame}


%% Diabetes: Mehr details, Typ 1 + 2
\begin{frame}
\frametitle{Diabetes Mellitus Typ 1 und 2}

\begin{center}
\includegraphics[width=\textwidth]{/home/melanie/Work/pictures/physiology/Diabetes1-2.jpg}
\end{center}


\end{frame}


%% Allgemeine Eigenschaften von Hormonen

\section{Allgemeine Eigenschaften von Hormonen}

\begin{frame}
\frametitle{Was sind Hormone?}

Hormone sind biochemische Botenstoffe. 

\pause

Sie können unterschieden werden nach:

\begin{itemize}
\item
Wirkungsradius
\item
Gewebe, in dem sie gebildet werden
\item
Chemischen Eigenschaften
\item
Zellulärer Wirkungsweise
\end{itemize}

\pause

Die Ausschüttung von Hormonen wird durch Regelkreise kontrolliert, bei denen oft negative Rückkopplungen eine Rolle spielen. \\

\pause

Der Hormonhaushalt kann aus mehreren Gründen gestört sein.

\end{frame}

%% %% autokrin - parakrin - endokrin
\begin{frame}
\frametitle{Hormone haben unterschiedliche Wirkungsradien}

\begin{center}
\includegraphics[width=\textwidth]{/home/melanie/Work/pictures/physiology/para_auto_endokrin.png}
\end{center}


\end{frame}


%% %%wo gebildet: Drüsen oder nicht
\begin{frame}
\frametitle{Hormone werden oft in spezialisierten Drüsen gebildet}

\begin{itemize}
\item
Glanduläre Hormone werden in speziellen Drüsen gebildet und daraus ausgeschüttet: Zirbeldrüse, Hypophyse, Schilddrüse, Thymus, Nebenniere, Pankreas, Gonaden
\item
Gewebshormone werden in anderen Geweben produziert (und wirken oft auf kürzere Distanzen) 
\end{itemize}


\end{frame}


%% %% Chemische Struktur
\begin{frame}
\frametitle{Hormone sind keine einheitliche chemische Klasse}

\begin{columns}[t]

\begin{column}{5cm}
\begin{block}{Peptid- und Proteohormone}

\begin{itemize}
\item
Derivate von Aminosäuren oder Polypeptide
\item
Meist wasserlöslich
\item
Binden an Rezeptoren an der Zelloberfläche (z.B. G-Protein gekoppelte Rezeptoren)
\item
Wirken schnell
\end{itemize}
\end{block}

\end{column}

\begin{column}{5cm}

\begin{block}{Steroidhormone}

\begin{itemize}
\item
Derivate von Cholesterin
\item
Fettlöslich
\item
Wirken im Inneren der Zelle (Genexpression)
\item
Wirken langsam
\end{itemize}
\end{block}

\end{column}


\end{columns}



\end{frame}



%% %% Regelkreis
\begin{frame}
\frametitle{Hormone werden von Regelkreisen kontrolliert}

\begin{center}
\includegraphics[width=0.8\textwidth]{/home/melanie/Work/pictures/physiology/Regelkreis_Biologie.png}
\end{center}

\pause

\textcolor{theme}{Wie sieht das bei Insulin aus?}


\end{frame}


%% %% Ursachen von Hormonstörungen
\begin{frame}
\frametitle{Hormonstörungen können viele Ursachen haben}

Beispiel Insulin/Glukagon oder Beispiel allgemeiner Regelkreis. An welchen Stellen kann es Probleme geben?

\pause

\begin{itemize}
\item
Zerstörung des Drüsengewebes
\item
Verminderte Ausschüttung des Hormons
\item
Erhöhte Ausschüttung des Hormons
\item
Probleme mit dem Transport ins Zielgewebe
\item
Veränderungen des Hormonrezeptors (Anzahl, Sensitivität)
\item
Veränderungen im Signalweg ``stromabwärts'' vom Rezeptor
\item
Veränderungen im Regelkreis (z.B. durch Veränderungen des Sensors)
\item
Weil Regelkreise häufig miteinander kommunizieren, kann eine Veränderung in einem Regelkreis eine Veränderung in einem zweiten Regelkreis nach sich ziehen.
\item
\dots

\end{itemize}

\end{frame}




\begin{frame}
\frametitle{Zusammenfassung \\ (und optionaler Selbsttest: Insulin/Glukagon)}

Hormone sind biochemische Botenstoffe. 

Sie können unterschieden werden nach:

\begin{itemize}
\item
Wirkungsradius
\item
Gewebe, in dem sie gebildet werden
\item
Chemischen Eigenschaften
\item
Zellulärer Wirkungsweise
\end{itemize}

Die Ausschüttung von Hormonen wird durch Regelkreise kontrolliert, bei denen oft negative Rückkopplungen eine Rolle spielen. \\

Der Hormonhaushalt kann aus mehreren Gründen gestört sein.

\end{frame}



\section{Hypothalamus und Hypophyse}

%% Überblick inkscape
\begin{frame}
\frametitle{Hypothalamus und Hypophyse sind an vielen Hormonwegen beteiligt}

\begin{center}
\includegraphics[width=\textwidth]{/home/melanie/Work/pictures/physiology/hormone.png}
\end{center}
\end{frame}


%% Zusammenspiel zwischen Hypothalamus und Hypophyse
\begin{frame}
\frametitle{Hypothalamus}

Der Hypothalamus produziert 

\begin{itemize}
\item
Hormone, die die Freisetzung anderer Hormone aus der Adenohypophyse (Hypophysenvorderlappen) kontrollieren:
\pause
\begin{itemize}
\item
Releasing Hormone (Liberine): fördern die Freisetzung
\item
Release-inhibiting Hormone (Statine): hemmen die Freisetzung
\end{itemize}
\pause
\item
Hormone die in der Neurohypophyse (Hypophysenhinterlappen) gespeichert und daraus freigesetzt werden: Oxytocin, Vasopressin (Antidiuretisches Hormon, ADH)
\end{itemize}


\end{frame}


%% Überblick funktionen, google image search mens magazine
\begin{frame}
\frametitle{Hypothalamus - Adenohypophyse - Gewebe}
\begin{center}
  \includegraphics[width=0.7\textwidth]{/home/melanie/Work/pictures/physiology/hypothalamus_hypophyse_gewebe.jpg}
\end{center}
\end{frame}



%%Fehlfunktionen

\begin{frame}
\frametitle{Hypophysäre Fehlfunktionen: Hormonüberschuss}

\begin{center}
\includegraphics[width=\textwidth]{/home/melanie/Work/pictures/physiology/hypophyse_hormonexzess.png}
\end{center}

\end{frame}


\begin{frame}
\frametitle{Hypophysäre Fehlfunktionen: Hormonverminderung}

\begin{center}
\includegraphics[width=\textwidth]{/home/melanie/Work/pictures/physiology/Hypophyse_Hormonverminderung.png}
\end{center}

\end{frame}


%% Heute aber nur
\begin{frame}
\frametitle{Wir konzentrieren uns heute auf 2 Bereiche:}

\begin{itemize}
\item
Hormone der Nebennierenrinde
\item
Hormone der Schilddrüse
\end{itemize}

\end{frame}

\section{Hormone der Nebennierenrinde}



%% Überblick

\begin{frame}
\frametitle{In der Nebennierenrinde werden Steroidhormone gebildet}

\begin{columns}[c]

\begin{column}{6cm}
\includegraphics[width=\textwidth]{/home/melanie/Work/pictures/physiology/Nebennierenrinde.png}
\end{column}

\begin{column}{5cm}

Die Nebennierenrinde bildet drei Arten von Steroidhormonen:
\begin{itemize}
\item
Glukokortikoide
\item
Mineralokortikoide
\item
Androgene
\end{itemize}

\pause

Steroidhormone sind fettlöslich. Sie diffundieren durch die Zellmembran und binden an Rezeptoren im Zellkern, um die Transkription bestimmter Gene zu regulieren. Dadurch wirken sie langsamer als Peptidhormone. 

\end{column}

\end{columns}

\end{frame}

%%%%%%%%%%%%%%%%%%&%%%%
%% compiled to here
%%%%%%%%%%%%%%%%%%&%%%%


%% Glukokortikoide
\begin{frame}
\frametitle{Glukokortikoide, z.B. Kortisol}

\begin{columns}[c]

\begin{column}{5cm}
\begin{center}
\includegraphics[width=\textwidth]{/home/melanie/Work/pictures/physiology/stress.jpg}
\end{center}

\end{column}

\begin{column}{5cm}

  
Freisetzung bei Stress: \\[0.2 cm]

\begin{tabular}{c}
CRH, ADH \\
 \(\downarrow\)\\
 ACTH \\
\(\downarrow\) \\
 Kortisol\\[0.5 cm]
\end{tabular}

\pause

Wirkung: Schnelle Bereitstellung von Energie (Flight-or-Fight response), Hemmung des Immunsystems (deshalb auch medizinisch eingesetzt, z.B. bei Autoimmunerkrankungen oder nach Organtransplantationen) \\

\end{column}

  

\end{columns}


\end{frame}



%% %% Glucokortikoide und negatives Feedback
\begin{frame}
\frametitle{Glukokortikoide: Regelkreis}

\begin{center}
\includegraphics[width=0.6\textwidth]{/home/melanie/Work/pictures/physiology/Glucocorticoids_Feedback_Loop.jpg}
\end{center}


\end{frame}



%% Erkrankungen
\begin{frame}
\frametitle{Glukokortikoidüberschuss}

\begin{columns}[c]

\begin{column}{4cm}
\begin{block}{Ursachen}
\begin{itemize}
\item
Verstärkte ACTH-Ausschüttung durch die Hypophyse (Morbus Cushing)
\item
Manche Tumoren
\item
Folge von Kortisoltherapie
\end{itemize}
\end{block}
\end{column}
\begin{column}{7cm}
\begin{center}

\includegraphics<1>[width=\textwidth]{/home/melanie/Work/pictures/physiology/glukokortikoid_ueberschuss_1.png}

\includegraphics<2>[width=\textwidth]{/home/melanie/Work/pictures/physiology/glukokortikoid_ueberschuss_2.png}

\includegraphics<3>[width=\textwidth]{/home/melanie/Work/pictures/physiology/glukokortikoid_ueberschuss_3.png}

\includegraphics<4>[width=\textwidth]{/home/melanie/Work/pictures/physiology/glukokortikoid_ueberschuss_4.png}

\includegraphics<5>[width=\textwidth]{/home/melanie/Work/pictures/physiology/glukokortikoid_ueberschuss_5.png}

\end{center}



\end{column}
\end{columns}



\end{frame}




%% Mineralokortikoide

\begin{frame}
\frametitle{Mineralokortikoide: Beispiel: Aldosteron}

\pause

\begin{columns}[c]
\begin{column}{3cm}

\begin{center}
 \includegraphics[width=\textwidth]{/home/melanie/Work/pictures/metaphore/wasserglas.jpg}
\end{center}

\end{column}

\begin{column}{7cm}
\begin{itemize}
\item
Bei Wassermangel wird antidiuretisches Hormon (ADH) aus der Neurohypophyse ausgeschüttet
\pause
\item
In der Nebennierenrinde bewirkt ADH Ausschüttung von Aldosteron
\pause
\item
Aldosteron fördert die Konservierung von Natrium in der Niere
\pause
\item
Dadurch wird Wasser zurückgehalten, das Blutvolumen steigt
\end{itemize}
\end{column}

\end{columns}

\end{frame}



%% \item
%% Androgene

\begin{frame}
\frametitle{Androgene}

\pause

\begin{itemize}
\item
ACTH löst die Produktion von DHEA (Dehydro-epi-andro-steron) in der Nebennierenrinde aus
\item 
Es gibt allerdings \emph{kein} negatives Feedback von DHEA zurück zur Aktivierung von ACTH!
\pause
\item
DHEA ist nicht selber aktiv sondern wird in Gonaden weiter verarbeitet: 
\end{itemize}

\begin{columns}[t]
\begin{column}{5cm}
\begin{block}{Im Hoden:}
Umwandlung von DHEA zu Testosteron\\
Umwandlung von Testosteron zu Dihydrotestosteron \\
\end{block}
\end{column}


\begin{column}{5cm}
\begin{block}{Im Ovar:}
Zusätzliche Produktion von DHEA \\
Umwandlung von DHEA zu Testosteron\\
Umwandlung von Testosteron in Östradiol \\
\end{block}
\end{column}
\end{columns}

\end{frame}


\section{Hormone der Schilddrüse}



%% Schilddrüse: Funktion



\begin{frame}
\frametitle{Die Schilddrüse ist wichtig für Energiestoffwechsel, Zellwachstum und Kalziumhaushalt}

\begin{columns}[c]
\begin{column}{5cm}
\begin{itemize}
\item
Bildung der \textcolor{theme}{Schilddrüsenhormone}:\\
 Triiodthyronin (T3) und Thyroxin (T4): \\
Energiestoffwechsel, Zellwachstum
\item
Bildung von Calcitonin: \\ Kalziumhaushalt und Knochenstabilität

\end{itemize}

\end{column}

\begin{column}{5cm}

\includegraphics[width=\textwidth]{/home/melanie/Work/pictures/physiology/Thyroid.jpg}

\end{column}

\end{columns}

\end{frame}






\begin{frame}
\frametitle{Chemische Eigenschaften von Schilddrüsenhormonen}



\begin{columns}[c]
\begin{column}{5cm}
\begin{center}
\includegraphics[width=\textwidth]{/home/melanie/Work/pictures/chemistry/Triiodthyronin.png}
\end{center}
\end{column}
\begin{column}{5cm}
Triiodthyronin (T3) und Thyroxin (T4) sind Tyrosin-Derivate mit 3 bzw. 4 gebundenen Iod-Atomen \\[0.2 cm]
\pause

Obwohl sie Derivate einer Aminosäure sind, verhalten sie sich ein bisschen wie Steroidhormone: Wirken an Rezeptoren im Inneren der Zelle und regulieren die Expression bestimmter Gene.  

\end{column}
\end{columns}

\end{frame}



%% %% %% Schilddrüse: Regelkreis
\begin{frame}
\frametitle{Schilddrüse: Regelkreis}

\begin{center}
\includegraphics[width=0.5\textwidth]{/home/melanie/Work/pictures/physiology/Schilddruese_Regelkreis.jpg}
\end{center}


\end{frame}



%% %% %% Erkrankungen

\begin{frame}
\frametitle{Erkrankungen}

Was könnten die Auswirkungen einer Schilddrüsenunterfunktion und -überfunktion sein?

\pause

\begin{columns}[t]
\begin{column}{5cm}
\begin{block}{Unterfunktion}
Eingeschränkte körperliche und geistige Entwicklung (bei Kindern)\\
Verlangsamter Stoffwechsel \\
Müdigkeit \\
Kälteempfindlichkeit \\
Verstopfung
\end{block}
\end{column}

\begin{column}{5cm}
\begin{block}{Überfunktion}
Erhöhte Körpertemperatur \\
Gesteigerte Herzfrequenz \\
Rastlosigkeit \\
Häufig Durchfall 
\end{block}
\end{column}

\end{columns}

\end{frame}


\begin{frame}
\frametitle{Hypophysäre Fehlfunktionen: Hormonüberschuss}

\begin{center}
\includegraphics[width=\textwidth]{/home/melanie/Work/pictures/physiology/hypophyse_hormonexzess.png}
\end{center}

\end{frame}


\begin{frame}
\frametitle{Hypophysäre Fehlfunktionen: Hormonverminderung}

\begin{center}
\includegraphics[width=\textwidth]{/home/melanie/Work/pictures/physiology/Hypophyse_Hormonverminderung.png}
\end{center}

\end{frame}


%% %% %% %% Review

\begin{frame}

\frametitle{Jetzt* sollten Sie folgendes können}



\begin{block}{Grundlagen:}
\begin{itemize}
\item
den allgemeinen Aufbau hormoneller Regelkreise erklären und Beispiele geben
\item
Hormone unterteilen 
\begin{itemize}
\item
 nach Wirkungsradius
\item 
nach chemischen Eigenschaften
\end{itemize}
\item
das hypothalamisch-hypophysäre System beschreiben und erklären
\item
Hormone des hypothalamisch-hypophysären Systems aufzählen und ihre Aktivierungsmuster beschreiben
\item
die Wirkung, chemischen Eigenschaften und Regulierung von T3 und T4 beschreiben.
\item
Hormone der Nebennierenrinde klassifizieren und ihre Regulierung und Wirkung beschreiben 
\item
Regulierung und Wirkung von Insulin und Glukagon erklären

\end{itemize}

\end{block}


\end{frame}



\begin{frame}

\frametitle{Jetzt* sollten Sie folgendes können}


\begin{block}{Klinik:}
\begin{itemize}
\item
Ursachen von Hormonstörungen benennen
\item
 Erkrankungen der Schilddrüse beschreiben und erklären
\item
 Morbus Cushing beschreiben und erklären
\item
Diabetes mellitus Typ 1 und 2 unterscheiden und erklären 
\end{itemize}

\end{block}

\end{frame}





%% %% %% %% Feedbackhinweisblock

\begin{frame}
\frametitle{Danke für Ihr Feedback!}

\begin{columns}[c]

\begin{column}{6cm}
\begin{center}
 \includegraphics[width=\textwidth]{/home/melanie/Work/pictures/metaphore/smilie_balloons.jpg}
\end{center}

\end{column}

\begin{column}{4cm}


\begin{center}
 \includegraphics[width=\textwidth]{feedback_QR.png}
\end{center}
\end{column}


\end{columns}

\end{frame}



%% %% %% Bildnachweis
\begin{frame}
\frametitle{Bildnachweis}

\begin{tiny}
 
\begin{itemize}

\item
Blutzuckermessgerät, Insulin-Spritze, Tabletten. Photo by \href{https://unsplash.com/@towfiqu999999?utm_source=unsplash&utm_medium=referral&utm_content=creditCopyText}{Towfiqu barbhuiya} \href{https://unsplash.com/s/photos/insulin?utm_source=unsplash&utm_medium=referral&utm_content=creditCopyText}{Unsplash}
  
\item
Diabetes Typ 1 und 2. BMBF-Broschüre ``Stoffwechselforschung''. 2010. Aufgerufen hier: \url{https://www.gesundheitsforschung-bmbf.de/de/jung-und-trotzdem-diabetiker.php}

\item
Glukagon. Truthortruth, CC BY-SA 3.0 \url{https://creativecommons.org/licenses/by-sa/3.0}, via Wikimedia Commons

\item
Häuserwand mit der Aufschrift ``How are you, really?'' Photo by \href{https://unsplash.com/@finnnyc?utm_source=unsplash&utm_medium=referral&utm_content=creditCopyText}{Finn} on \href{https://unsplash.com/s/photos/hormones?utm_source=unsplash&utm_medium=referral&utm_content=creditCopyText}{Unsplash}
  

\item
Hormone der endokrinen Drüsen im Nervensystem. Bearbeitung (deutsche Übersetzung) der folgenden Abbildung: By LadyofHats - made myself based on the information found on the wikipedia article, Public Domain, \url{https://commons.wikimedia.org/w/index.php?curid=11541118}

\item
Hormone der Nebennierenrinde. Aus: Florian Lang, Michael Föller. Nebennierenrindenhormone. Springer-Verlag GmbH Deutschland, ein Teil von Springer Nature 2019. In: R. Brandes et al. (Hrsg.), Physiologie des Menschen, Springer-Lehrbuch \url{https://doi.org/10.1007/978-3-662-56468-4_77}


\item
Hypophysäre Fehlfunktionen. Arastéh K, Baenkler H, Bieber C, Brandt R, Chatterjee T, Dill T, Ditting T, Duckert M, Eich W et al., ed.  Duale Reihe Innere Medizin. 4., überarbeitete Auflage. Stuttgart: Thieme; 2018. doi:10.1055/b-005-145255

\item
Insulin. Von Anmoll - Eigenes Werk mittels: BALLView, CC BY-SA 2.0 de, \url{https://commons.wikimedia.org/w/index.php?curid=9021528}

\item
Insulinausschüttung. Der ursprünglich hochladende Benutzer war Prisonblues in der Wikipedia auf Englisch - Übertragen aus en.wikipedia nach Commons., CC BY-SA 3.0, \url{https://commons.wikimedia.org/w/index.php?curid=1538766}

\item
Kortisol. Von NEUROtiker - Eigenes Werk, Gemeinfrei, \url{https://commons.wikimedia.org/w/index.php?curid=2319372}

%% all lectures
\item
Logo der MSB. MSB Medical School Berlin, Public Domain, via Wikimedia Commons
%%%%%%%%%%%%


%% all lectures
\item
Luftballons mit frohen und traurigen Smilies. Photo by \href{https://unsplash.com/@artbyhybrid?utm_source=unsplash&utm_medium=referral&utm_content=creditCopyText}{Hybrid} on \href{https://unsplash.com/s/photos/feedback?utm_source=unsplash&utm_medium=referral&utm_content=creditCopyText}{Unsplash}
%%%%%%%%%%%
\end{itemize}
\end{tiny}
\end{frame}

\begin{frame}
\frametitle{Bildnachweis}

\begin{tiny}
\begin{itemize}

\item
Parakrine, autokrine und endokrine Hormone. Eigenes Werk, CC BY-SA 4.0, 2022.

\item
Regelkreis allgemein. Von Sciencia58 - Eigenes Werk, CC BY-SA 4.0, \url{https://commons.wikimedia.org/w/index.php?curid=96633159}


\item
Regelkreis der Glucocorticoide. OpenStax College, CC BY 3.0 \url{https://creativecommons.org/licenses/by/3.0}, via Wikimedia Commons

\item
Regelkreis der Schilddrüsenhormone. Von Geo-Science-International - Eigenes Werk, CC BY-SA 4.0, \url{https://commons.wikimedia.org/w/index.php?curid=47357720}



\item
Regulierung von Insulin und Glukagon. Aus: Florian Lang, Michael Föller. Pankreashormone. Springer-Verlag GmbH Deutschland, ein Teil von Springer Nature 2019 In: R. Brandes et al. (Hrsg.), Physiologie des Menschen, Springer-Lehrbuch \url{https://doi.org/10.1007/978-3-662-56468-4_76}


\item
Schilddrüse. CFCF, CC BY-SA 4.0 \url{https://creativecommons.org/licenses/by-sa/4.0}, via Wikimedia Commons

\item
Signalwege Hypothalamus - Hypophysenvorderlappen - Gewebe - Funktion.  Aus: Schöfl C. Neuroendokrine Dysfunktion nach Schädelhirntrauma und Subarachnoidalblutung. \emph{Blickpunkt der Mann} 2008; 6 (Sonderheft 1): 22-24

\item
Stress. Photo by \href{https://unsplash.com/@nublson?utm_source=unsplash&utm_medium=referral&utm_content=creditCopyText}{Nubelson Fernandes} on \href{https://unsplash.com/s/photos/stress?utm_source=unsplash&utm_medium=referral&utm_content=creditCopyText}{Unsplash}

\item
Sü\ss igkeiten. Photo by \href{https://unsplash.com/es/@fluffmedia?utm_source=unsplash&utm_medium=referral&utm_content=creditCopyText}{Deidre Schlabs} on \href{https://unsplash.com/s/photos/sweets?utm_source=unsplash&utm_medium=referral&utm_content=creditCopyText}{Unsplash}
  

\item
Triiodthyronin. Von Jü - Eigenes Werk, Gemeinfrei, \url{https://commons.wikimedia.org/w/index.php?curid=39939375}


\item
Wasserglas. Photo by \href{https://unsplash.com/@nicoruiz01981?utm_source=unsplash&utm_medium=referral&utm_content=creditCopyText}{Nicolas Ruiz} on \href{https://unsplash.com/s/photos/water-glass?utm_source=unsplash&utm_medium=referral&utm_content=creditCopyText}{Unsplash}
  

\item

Wirkung von Glukagon. Von Yikrazuul - Eigenes Werk, CC BY-SA 3.0, \url{https://commons.wikimedia.org/w/index.php?curid=9532952}

\item
Wirkung von Insulin - Meine eigene Arbeit, mit BioRender, 2022.

\end{itemize}
\end{tiny}
\end{frame}



\end{document}
